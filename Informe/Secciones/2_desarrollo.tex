
\subsection{Configuración de los routers}

Se verificó el archivo de parámetros de configuración \texttt{/crypto/conf/config.sh} y el contenido se expone a continuación

%%% Acá se ponen el contenido de config.sh %%%
\lstset{language=bash}
\begin{lstlisting}
	ls /etc/vim
\end{lstlisting}

Luego se procede a configurar los routers \textsc{R1} y \textsc{R2}.

Se corren los comandos:

\begin{lstlisting}
	hostname R1
	ifconfig eth0 192.168.NRO_GR.41 netmask 255.255.255.0
	ifconfig eth1 10.NRO_GR.1.1 netmask 255.255.255.0
\end{lstlisting}

\begin{lstlisting}
	hostname R2
	ifconfig eth0 192.168.NRO_GR.42 netmask 255.255.255.0
	ifconfig eth1 10.NRO_GR.2.1 netmask 255.255.255.0
\end{lstlisting}

Y se modifica el archivo \texttt{/etc/hosts}:
\begin{lstlisting}
	127.0.0.1 localhost
	192.168.NRO_GR.41 R1
	192.168.NRO_GR.42 R2
\end{lstlisting}

\subsection{Configuración de los host}

\subsection{Verificación de la red}

\subsection{Generación de claves IPSEC}

\subsection{Obtención de las claves remotas}

\subsection{Configuración del IPSEC}

\subsection{Inicio del enlace IPSEC}

\subsection{Verificación del túnel}

\subsection{Captura del protocolo}

\subsection{Desencriptación del tráfico}
